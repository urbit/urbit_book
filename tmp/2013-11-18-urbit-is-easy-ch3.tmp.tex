\chapter{Nock Is Easy}

\emph{``You get used to it. I don’t even see the code. All I see is blonde, brunette, redhead.''}
\textbf{(The Matrix)}

\section{Fundamentals}

Now that we have all the tools, let's learn Nock from scratch.
Here are all the rules defining \kode{*}:

\begin{framed_shaded}
\begin{Verbatim}[fontsize=\relsize{-2.5},fontseries=b,commandchars=\\\{\}]
23 ::    *[a [b c] d]     [*[a b c] *[a d]]
24 ::
25 ::    *[a 0 b]         /[b a]
26 ::    *[a 1 b]         b
27 ::    *[a 2 b c]       *[*[a b] *[a c]]
28 ::    *[a 3 b]         ?*[a b]
29 ::    *[a 4 b]         +*[a b]
30 ::    *[a 5 b]         =*[a b]
31 ::
32 ::    *[a 6 b c d]     *[a 2 [0 1] 2 [1 c d] [1 0] 2 [1 2 3] [1 0] 4 4 b]
33 ::    *[a 7 b c]       *[a 2 b 1 c]
34 ::    *[a 8 b c]       *[a 7 [[7 [0 1] b] 0 1] c]
35 ::    *[a 9 b c]       *[a 7 c 2 [0 1] 0 b]
36 ::    *[a 10 [b c] d]  *[a 8 c 7 [0 3] d]
37 ::    *[a 10 b c]      *[a c]
38 ::
39 ::    *a               *a
\end{Verbatim}
\end{framed_shaded}
As we saw in the last chapter, when \kode{a} is an atom, \kode{*a} is
always an error.  So Nock proper is a function of a cell.
Informally, that cell is always described as a pair

\begin{framed_shaded}
\begin{Verbatim}[fontsize=\relsize{-2.5},fontseries=b,commandchars=\\\{\}]
[subject formula]
\end{Verbatim}
\end{framed_shaded}
where \kode{subject} is the data and \kode{formula} is the program.  Notice
that \kode{a} in the rules above, except 39, is always the subject.
So: let's learn how to write a Nock formula.

\section{Autocons}

We observe from the rules above that a formula, too, is always a
cell.  But when we look inside that cell, we see two basic kinds
of formulas:

\begin{framed_shaded}
\begin{Verbatim}[fontsize=\relsize{-2.5},fontseries=b,commandchars=\\\{\}]
[operator operands]
[formula formula]
\end{Verbatim}
\end{framed_shaded}
An operator is always an atom (\kode{0} through \kode{10}).  A formula is
always a cell.  Line 23 distinguishes these forms:

\begin{framed_shaded}
\begin{Verbatim}[fontsize=\relsize{-2.5},fontseries=b,commandchars=\\\{\}]
23 ::    *[a [b c] d]     [*[a b c] *[a d]]
\end{Verbatim}
\end{framed_shaded}
Suppose you have two formulas \kode{f} and \kode{g}, each of which computes
some function of the subject \kode{s}.  You can then construct the
formula \kode{h} as \kode{[f g]}; and \kode{h(s)} equals \kode{[f(s) g(s)]}.

For instance, recall our computation from the last chapter:

\begin{framed_shaded}
\begin{Verbatim}[fontsize=\relsize{-2.5},fontseries=b,commandchars=\\\{\}]
*[[19 42] [0 3] 0 2]
\end{Verbatim}
\end{framed_shaded}
\kode{s} is \kode{[19 42]}, \kode{f} is \kode{[0 3]}, \kode{g} is \kode{[0 2]}, \kode{h} is \kode{[[0 3] 0
2]}.  \kode{f(s)} is \kode{42}; \kode{g(s)} is \kode{19}; so \kode{h(s)} is \kode{[42 19]}.

Why?  We could have an operator \kode{11}, \kode{cons} to a Lisp veteran,
with the operands \kode{f} and \kode{g}---so instead of writing

\begin{framed_shaded}
\begin{Verbatim}[fontsize=\relsize{-2.5},fontseries=b,commandchars=\\\{\}]
[[0 3] 0 2]
\end{Verbatim}
\end{framed_shaded}
we'd say

\begin{framed_shaded}
\begin{Verbatim}[fontsize=\relsize{-2.5},fontseries=b,commandchars=\\\{\}]
[11 [0 3] 0 2]
\end{Verbatim}
\end{framed_shaded}
But not only is this less elegant, it's less convenient.  Of
course, convenience at the Nock level matters little, but we
repeat this pattern at the Hoon level---where it's often more
pleasant to say \kode{[a b]} than \kode{(cons a b)}.

\section{Basic operators}

Nock is small, but it could be smaller.  If we didn't care at all
about the efficiency of the interpreter---in other words, if Nock
was a theoretical exercise rather than a practical tool---we
could make do with just the first six operators:

\begin{framed_shaded}
\begin{Verbatim}[fontsize=\relsize{-2.5},fontseries=b,commandchars=\\\{\}]
25 ::    *[a 0 b]         /[b a]
26 ::    *[a 1 b]         b
27 ::    *[a 2 b c]       *[*[a b] *[a c]]
28 ::    *[a 3 b]         ?*[a b]
29 ::    *[a 4 b]         +*[a b]
30 ::    *[a 5 b]         =*[a b]
\end{Verbatim}
\end{framed_shaded}
Let's run through them one by one.

\subsection{\kode{0}}

\kode{0} just applies the \kode{/} function:

\begin{framed_shaded}
\begin{Verbatim}[fontsize=\relsize{-2.5},fontseries=b,commandchars=\\\{\}]
25 ::    *[a 0 b]         /[b a]
\end{Verbatim}
\end{framed_shaded}
For any subject \kode{a}, the formula \kode{[0 b]} produces \kode{/[b a]}, which
is why

\begin{framed_shaded}
\begin{Verbatim}[fontsize=\relsize{-2.5},fontseries=b,commandchars=\\\{\}]
*[[19 42] 0 3]
\end{Verbatim}
\end{framed_shaded}
is \kode{/[3 19 42]}, which is \kode{42}.

\subsection{\kode{1}}

\kode{1} just ignores its subject and produces its operand:

\begin{framed_shaded}
\begin{Verbatim}[fontsize=\relsize{-2.5},fontseries=b,commandchars=\\\{\}]
26 ::    *[a 1 b]         b
\end{Verbatim}
\end{framed_shaded}

\subsection{\kode{2}}

\kode{2} is the only interesting basic operator:

\begin{framed_shaded}
\begin{Verbatim}[fontsize=\relsize{-2.5},fontseries=b,commandchars=\\\{\}]
27 ::    *[a 2 b c]       *[*[a b] *[a c]]
\end{Verbatim}
\end{framed_shaded}
Here we generate a calculation to perform.  Given the formula \kode{[2
b c]}, \kode{b} is a formula for generating the new subject; \kode{c} is a
formula for generating the new formula.  To compute \kode{*[a 2 b c]},
we evaluate both \kode{b} and \kode{c} against the current subject \kode{a}.

\subsection{\kode{3}, \kode{4}, \kode{5}}

\kode{3}, \kode{4}, and \kode{5} just apply \kode{?}, \kode{+} and \kode{=} respectively---that is, cell/atom, increment, and equals.

\begin{framed_shaded}
\begin{Verbatim}[fontsize=\relsize{-2.5},fontseries=b,commandchars=\\\{\}]
28 ::    *[a 3 b]         ?*[a b]
29 ::    *[a 4 b]         +*[a b]
30 ::    *[a 5 b]         =*[a b]
\end{Verbatim}
\end{framed_shaded}

\section{Macros}

Operators \kode{6} through \kode{10} are all essentially macros:

\begin{framed_shaded}
\begin{Verbatim}[fontsize=\relsize{-2.5},fontseries=b,commandchars=\\\{\}]
32 ::    *[a 6 b c d]     *[a 2 [0 1] 2 [1 c d] [1 0] 2 [1 2 3] [1 0] 4 4 b]
33 ::    *[a 7 b c]       *[a 2 b 1 c]
34 ::    *[a 8 b c]       *[a 7 [[7 [0 1] b] 0 1] c]
35 ::    *[a 9 b c]       *[a 7 c 2 [0 1] 0 b]
36 ::    *[a 10 [b c] d]  *[a 8 c 7 [0 3] d]
37 ::    *[a 10 b c]      *[a c]
\end{Verbatim}
\end{framed_shaded}
Each of these cases just resolves to another Nock computation, in
which each pattern matched on the left appears no more than once
on the right.  I.e., it's a macro.  But what do the macros do?
Let's work through them, from easiest to hardest.

\subsection{\kode{10} (37)}

The second case of 10 is so easy it's puzzling:

\begin{framed_shaded}
\begin{Verbatim}[fontsize=\relsize{-2.5},fontseries=b,commandchars=\\\{\}]
37 ::    *[a 10 b c]      *[a c]
\end{Verbatim}
\end{framed_shaded}
For any \kode{b}, the formula \kode{[10 b c]} seems to be perfectly
equivalent to the formula \kode{c}.  But why?  Why would we say
\kode{[10 b c]} when we could just say \kode{c}?

The answer is that \kode{10} is a hint to the interpreter.  It's true
that \kode{[10 b c]} has to be \emph{semantically} equivalent to \kode{c}, but
it doesn't have to be \emph{practically} equivalent.  Since whatever
information is in \kode{b} is discarded, a practical interpreter is
free to ignore it, or to use it in any way that does not affect
the results of the computation.

\subsection{\kode{7}}

\kode{7} is our next easiest macro:

\begin{framed_shaded}
\begin{Verbatim}[fontsize=\relsize{-2.5},fontseries=b,commandchars=\\\{\}]
33 ::    *[a 7 b c]       *[a 2 b 1 c]
\end{Verbatim}
\end{framed_shaded}
Informally, the formula \kode{[7 b c]} composes the formulas \kode{b} and
\kode{c}.  To use a bit of math notation, if \kode{d} is \kode{[7 b c]},

\begin{framed_shaded}
\begin{Verbatim}[fontsize=\relsize{-2.5},fontseries=b,commandchars=\\\{\}]
d(a) == c(b(a))
\end{Verbatim}
\end{framed_shaded}
Let's see how this works by applying some reductions to the
definition of \kode{7}, and producing a simpler definition that
doesn't look like a macro:

\begin{framed_shaded}
\begin{Verbatim}[fontsize=\relsize{-2.5},fontseries=b,commandchars=\\\{\}]
*[a 2 b 1 c]

    \PYZlt{}\PYZlt{}27 ::    *[a 2 b c]       *[*[a b] *[a c]]\PYZgt{}\PYZgt{}

*[*[a b] *[a 1 c]]

    \PYZlt{}\PYZlt{}26 ::    *[a 1 b]         b\PYZgt{}\PYZgt{}

*[*[a b] c]
\end{Verbatim}
\end{framed_shaded}
So we can write a revised line 33, perhaps slightly clearer:

\begin{framed_shaded}
\begin{Verbatim}[fontsize=\relsize{-2.5},fontseries=b,commandchars=\\\{\}]
33r::    *[a 7 b c]       *[*[a b] c]
\end{Verbatim}
\end{framed_shaded}

\subsection{\kode{8}}

\kode{8} looks slightly horrible but you shouldn't fear it at all:

\begin{framed_shaded}
\begin{Verbatim}[fontsize=\relsize{-2.5},fontseries=b,commandchars=\\\{\}]
34 ::    *[a 8 b c]       *[a 7 [[7 [0 1] b] 0 1] c]
\end{Verbatim}
\end{framed_shaded}
What does this even mean?  Let's go through the same process
of reducing it.

\begin{framed_shaded}
\begin{Verbatim}[fontsize=\relsize{-2.5},fontseries=b,commandchars=\\\{\}]
*[a 7 [[7 [0 1] b] 0 1] c]

  \PYZlt{}\PYZlt{}33r::    *[a 7 b c]       *[*[a b] c]\PYZgt{}\PYZgt{}

*[*[a [7 [0 1] b] 0 1] c]

  \PYZlt{}\PYZlt{}23 ::    *[a [b c] d]     [*[a b c] *[a d]]\PYZgt{}\PYZgt{}

*[[*[a 7 [0 1] b] *[a 0 1]] c]

  \PYZlt{}\PYZlt{}33r::    *[a 7 b c]       *[*[a b] c]\PYZgt{}\PYZgt{}

*[[*[*[a 0 1] b] *[a 0 1]] c]

  \PYZlt{}\PYZlt{}25 ::    *[a 0 b]         /[b a]\PYZgt{}\PYZgt{}

*[[*[a b] a] c]
\end{Verbatim}
\end{framed_shaded}
So our revised rule 34:

\begin{framed_shaded}
\begin{Verbatim}[fontsize=\relsize{-2.5},fontseries=b,commandchars=\\\{\}]
34r::    *[a 8 b c]       *[[*[a b] a] c]
\end{Verbatim}
\end{framed_shaded}
What does this actually do?  Well, look at it.  It evaluates the
formula \kode{c} with the cell of \kode{*[a b]} and the original subject
\kode{a}.  In other words, in math notation, if \kode{d} is \kode{[8 b c]},

\begin{framed_shaded}
\begin{Verbatim}[fontsize=\relsize{-2.5},fontseries=b,commandchars=\\\{\}]
d(a) == c([b(a) a])
\end{Verbatim}
\end{framed_shaded}
But why?  Suppose, for the purposes of \kode{c}, we need not just \kode{a},
but some intermediate noun computed from \kode{a} that will be useful
in \kode{c}'s calculation.  We apply \kode{c} with a new subject that's a
cell of the intermediate value and the old subject---not at all
unlike pushing a new variable on the stack.

For extra credit, a good question to ask yourself: why do we
need to write \kode{[7 [0 1] b]} and not just \kode{b}?

\subsection{\kode{10} (36)}

We now understand all the moving parts we need to figure out the
other reduction of \kode{10}:

\begin{framed_shaded}
\begin{Verbatim}[fontsize=\relsize{-2.5},fontseries=b,commandchars=\\\{\}]
36 ::    *[a 10 [b c] d]  *[a 8 c 7 [0 3] d]
\end{Verbatim}
\end{framed_shaded}
Reducing:

\begin{framed_shaded}
\begin{Verbatim}[fontsize=\relsize{-2.5},fontseries=b,commandchars=\\\{\}]
*[a 8 c 7 [0 3] d]

  \PYZlt{}\PYZlt{}34r::    *[a 8 b c]       *[[*[a b] a] c]\PYZgt{}\PYZgt{}

*[[*[a c] a] [7 [0 3] d]]

  \PYZlt{}\PYZlt{}33r::    *[a 7 b c]       *[*[a b] c]\PYZgt{}\PYZgt{}

*[*[[*[a c] a] 0 3] d]
\end{Verbatim}
\end{framed_shaded}
If you've assimilated a bit of Nock already, you may feel the
temptation to reduce this to

\begin{framed_shaded}
\begin{Verbatim}[fontsize=\relsize{-2.5},fontseries=b,commandchars=\\\{\}]
*[a d]
\end{Verbatim}
\end{framed_shaded}
since it would be very reasonable to think that

\begin{framed_shaded}
\begin{Verbatim}[fontsize=\relsize{-2.5},fontseries=b,commandchars=\\\{\}]
*[[*[a c] a] 0 3]
\end{Verbatim}
\end{framed_shaded}
is just \kode{a}.  And it seems to be---given the semantics of 8 as
we've explained them.

But there's a problem, which is that \kode{c} might not terminate.
If \kode{c} terminates, this reduction is correct.  Otherwise it's not. 
So the best we can do is:

\begin{framed_shaded}
\begin{Verbatim}[fontsize=\relsize{-2.5},fontseries=b,commandchars=\\\{\}]
36r::    *[a 10 [b c] d]  *[*[[*[a c] a] 0 3] d]
\end{Verbatim}
\end{framed_shaded}
Why?  \kode{10} in either case is a hint.  If \kode{x} in \kode{[10 x y]} is an
atom, we reduce line 37 and \kode{x} is simply discarded.  Otherwise,
\kode{x} is a cell \kode{[b c]}; \kode{b} is discarded, but \kode{c} is computed as a
formula and its result is discarded.

Effectively, this mechanism lets us feed both static and dynamic
information into the interpreter's hint mechanism.

\subsection{\kode{6}}

\kode{6} certainly looks intimidating:

\begin{framed_shaded}
\begin{Verbatim}[fontsize=\relsize{-2.5},fontseries=b,commandchars=\\\{\}]
32 ::    *[a 6 b c d]     *[a 2 [0 1] 2 [1 c d] [1 0] 2 [1 2 3] [1 0] 4 4 b]
\end{Verbatim}
\end{framed_shaded}
We could explain \kode{6} as a reduction sequence.  But it's a long
one.  Instead, let's invent another operator which makes \kode{6} easy:

\begin{framed_shaded}
\begin{Verbatim}[fontsize=\relsize{-2.5},fontseries=b,commandchars=\\\{\}]
   ::    \PYZdl{}[0 b c]         b
   ::    \PYZdl{}[1 b c]         c
\end{Verbatim}
\end{framed_shaded}
Then we can restate \kode{6} quite compactly:

\begin{framed_shaded}
\begin{Verbatim}[fontsize=\relsize{-2.5},fontseries=b,commandchars=\\\{\}]
32r::    *[a 6 b c d]     *[a \PYZdl{}[*[a b] c d]]
\end{Verbatim}
\end{framed_shaded}
\kode{6} stands revealed as the humble if-then-else.  Nock \emph{is} easy.

This excuse for an explanation may not satisfy everyone.  A good
exercise is to check that \kode{6} as defined \emph{actually} has these
properties---and can't be simplified.

\subsection{\kode{9}}

\kode{9} is an audacious mystery:

\begin{framed_shaded}
\begin{Verbatim}[fontsize=\relsize{-2.5},fontseries=b,commandchars=\\\{\}]
35 ::    *[a 9 b c]       *[a 7 c 2 [0 1] 0 b]
\end{Verbatim}
\end{framed_shaded}
We'll reduce \kode{9} but not explain it.  When we use it in an
example, it'll be obvious what it is.

\begin{framed_shaded}
\begin{Verbatim}[fontsize=\relsize{-2.5},fontseries=b,commandchars=\\\{\}]
*[a 7 c 2 [0 1] 0 b]]

  \PYZlt{}\PYZlt{}33r::    *[a 7 b c]       *[*[a b] c]\PYZgt{}\PYZgt{}

*[*[a c] 2 [0 1] 0 b]]
	
  \PYZlt{}\PYZlt{}27 ::    *[a 2 b c]       *[*[a b] *[a c]]\PYZgt{}\PYZgt{}

*[*[*[a c] [0 1]] *[*[a c] 0 b]]

  \PYZlt{}\PYZlt{}25 ::    *[a 0 b]         /[b a]\PYZgt{}\PYZgt{}

*[*[a c] *[*[a c] 0 b]]
\end{Verbatim}
\end{framed_shaded}
So we have:

\begin{framed_shaded}
\begin{Verbatim}[fontsize=\relsize{-2.5},fontseries=b,commandchars=\\\{\}]
35r::    *[a 9 b c]       *[*[a c] *[*[a c] 0 b]]
\end{Verbatim}
\end{framed_shaded}
If you have a really fine instinctive sense of Nock, you might
understand what \kode{9} is for.  Otherwise, don't worry for now.