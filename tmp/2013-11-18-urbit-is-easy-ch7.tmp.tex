\chapter{Hoon Computes}

\emph{``I've only been in love with a beer bottle and a mirror.''}
\textbf{(Sid Vicious)}

\section{Syntax}

Is Hoon actually an improvement on Nock?  The jury holds out.
Here is our old decrement function, from the Nock battery:

\begin{framed_shaded}
\begin{Verbatim}[fontsize=\relsize{-2.5},fontseries=b,commandchars=\\\{\}]
[ 8
  [1 0]
  [ 8
    [ 1
      [ 6
        [5 [4 0 6] [0 30]]
        [0 6]
        [9 2 [0 2] [4 0 6] [0 7]]
      ]
    ]
    [9 2 0 1]
  ]
]
\end{Verbatim}
\end{framed_shaded}
This clearly it is not for the little boys and girls.  But wait---is Hoon any less formidable?  The equivalent Hoon twig:

\begin{framed_shaded}
\begin{Verbatim}[fontsize=\relsize{-2.5},fontseries=b,commandchars=\\\{\}]
|=  a=@
=|  b=@
|\PYZhy{}  ?:  =(a +(b))
      b
    \PYZdl{}(b +(b))
\end{Verbatim}
\end{framed_shaded}
Whaa?  We can also write the exact same decrement as:

\begin{framed_shaded}
\begin{Verbatim}[fontsize=\relsize{-2.5},fontseries=b,commandchars=\\\{\}]
|=(a=@ =|(b=@ |\PYZhy{}(?:(=(a +(b)) b \PYZdl{}(b +(b))))))
\end{Verbatim}
\end{framed_shaded}
for instance:

\begin{framed_shaded}
\begin{Verbatim}[fontsize=\relsize{-2.5},fontseries=b,commandchars=\\\{\}]
\PYZti{}waclux\PYZhy{}tomwyc/try=\PYZgt{} (|=(a=@ =|(b=@ |\PYZhy{}(?:(=(a +(b)) b \PYZdl{}(b +(b)))))) 42)
41
\end{Verbatim}
\end{framed_shaded}
or even more cryptically, 

\begin{framed_shaded}
\begin{Verbatim}[fontsize=\relsize{-2.5},fontseries=b,commandchars=\\\{\}]
\PYZti{}waclux\PYZhy{}tomwyc/try=\PYZgt{} \PYZpc{}.(42 |=(a=@ =|(b=@ |\PYZhy{}(?:(=(a +(b)) b \PYZdl{}(b +(b)))))))
41
\end{Verbatim}
\end{framed_shaded}
And I showed this to my daughter, who ran away in tears.  Dear
reader, I hope you're not tempted to do the same.

Hoon actually is easy!  It's much easier than Nock.  Nock really
isn't that hard itself, unless you want to do something useful
with it.  But to learn Hoon, you just have to accept that it's
really quite orthogonal to any or all of those \emph{20th-century}
languages you knew already.  Even the functional ones, mostly.
If anything, this knowledge makes it harder to learn Hoon.

Our mutual hope is that by the time you do know Hoon, you will
simply be able to \emph{see} a twig like

\begin{framed_shaded}
\begin{Verbatim}[fontsize=\relsize{-2.5},fontseries=b,commandchars=\\\{\}]
|=  a=@
=|  b=@
|\PYZhy{}  ?:  =(a +(b))
      b
    \PYZdl{}(b +(b))
\end{Verbatim}
\end{framed_shaded}
and, far from merely able to follow this code, \emph{actually observe
in your mind's eye the function itself}---not unlike Keanu with
kung fu.  You will simply look at this strange collection of
squiggles and \emph{see decrement}.

(Our feeling is that the use of reserved words in most languages,
by instead activating the verbal lobes, disrupts this sense of
directly, or at least visually, perceiving the program itself.
If you object that this sounds too much like Science, Science
implies testing hypotheses whereas we generally just wing it.)

Think of learning Hoon as learning to program all over again.  If
nothing else, it's a sort of eccentric adventure sport.  Or even
a mystery---can a language be esoteric, yet useful?  Click here
to find out more.

\subsection{Glyphs}

It's actually worse than that---learning Hoon is learning to
\emph{read} all over again.  Again, Hoon is a reserved-word-free
language---any text in the program is part of the program.

We use so many of these ASCII glyphs that we like to be able
to read them out loud.  A language is meant to be \emph{said}.  The
squiggles have conventional names, sort of, some of them easy to
say, others not so much.  So we've renamed them:

\begin{framed_shaded}
\begin{Verbatim}[fontsize=\relsize{-2.5},fontseries=b,commandchars=\\\{\}]
ace  space      gal  \PYZlt{}          per  )
bar  |          gar  \PYZgt{}          sel  [
bas  \PYZbs{}          hax  \PYZsh{}          sem  ;
buc  \PYZdl{}          hep  \PYZhy{}          ser  ]
cab  \PYZus{}          kel  \PYZob{}          sig  \PYZti{}
cen  \PYZpc{}          ker  \PYZcb{}          soq  \PYZsq{}
col  :          ket  \PYZca{}          tar  *
com  ,          lus  +          tec  `
doq  \PYZdq{}          pam  \PYZam{}          tis  =
dot  .          pat  @          wut  ?
fas  /          pel  (          zap  !
\end{Verbatim}
\end{framed_shaded}
You just have to memorize these names.  Sorry.  We accept
that they are vile, barbaric and loathsome.  So is life.

\subsection{Runes}

But is this at least enough symbols?  Alas, nowhere near.
ASCII's glyph supply is not the greatest, but we can make all the
squiggles we need by forming digraphs, or \emph{runes}.  For example:
\kode{bartis}, i.e., \kode{\textbar{}=}.  

To pronounce a rune, concatenate the glyph names, stressing the
first syllable and softening the second vowel into a ``schwa.''
Hence, to say \kode{\sig .}, say ``sigdot.''  To say \kode{\textbar{}=}, say ``bartis.''
Which has an inevitable tendency to turn into ``barts''---a sin
to be encouraged.  In any language actually spoken by actual
humans, benign indolence soon rounds off any rough edges.

There are a few runes with irregular special pronunciations:

\begin{framed_shaded}
\begin{Verbatim}[fontsize=\relsize{-2.5},fontseries=b,commandchars=\\\{\}]
\PYZhy{}\PYZhy{}    hephep    phep    
+\PYZhy{}    lushep    slep
++    luslus    slus
==    tistis    stet
+\PYZlt{}    lusgal    glus
+\PYZgt{}    lusgar    gras
\PYZhy{}\PYZlt{}    hepgal    gelp
\PYZhy{}\PYZgt{}    hepgar    garp     
\end{Verbatim}
\end{framed_shaded}
Let's look at that decrement twig again:

\begin{framed_shaded}
\begin{Verbatim}[fontsize=\relsize{-2.5},fontseries=b,commandchars=\\\{\}]
|=  a=@
=|  b=@
|\PYZhy{}  ?:  =(a +(b))
      b
    \PYZdl{}(b +(b))
\end{Verbatim}
\end{framed_shaded}
If we had to read this twig, omitting the spaces (which only a
real purist would pronounce), we'd say: ``bartis A is pat tisbar B
is pat barhep wutcol tis pel A lus pel B perper B bucpel B luspel B
perper.'' The authorities would then arrive, and drag us out in a
big net. Definitely don't do this at the airport.

Hoon has almost 90 digraphic runes.  Worse, ``Hoon runes'' are
inevitably shortened to ``hoons''---a ridiculous non-English word
due originally to Wallace Stevens, which also has the unique
property of reducing Australians to convulsions.

None of this should scare you.  First, 90 symbols is not a lot
compared to, say, Chinese.  Second, hoons are easier than you'd
expect to organize in your head, because the choice of glyph is
not random.  Third, no one lives in Australia and nobody cares.

The second glyph in a hoon means little or nothing, but the first
defines a rough semantic category.   These categories are:

\begin{framed_shaded}
\begin{Verbatim}[fontsize=\relsize{-2.5},fontseries=b,commandchars=\\\{\}]
|  bar    core construction
\PYZdl{}  buc    tiles and tiling
\PYZpc{}  cen    invocations
:  col    tuples
.  dot    nock operators
\PYZca{}  ket    type conversions
;  sem    miscellaneous macros
\PYZti{}  sig    hints
=  tis    compositions
?  wut    conditionals, booleans, tests
!  zap    special operations
\end{Verbatim}
\end{framed_shaded}

\subsection{Regular forms}

Like natural languages, Hoon wraps a blanket of funky, irregular,
easy-to-type abbreviations around a comforting but verbose
regular core.  Let's explain the regular core first.

Consider a very simple twig in completely regular form:

\begin{framed_shaded}
\begin{Verbatim}[fontsize=\relsize{-2.5},fontseries=b,commandchars=\\\{\}]
?:  \PYZam{}
  47
52
\end{Verbatim}
\end{framed_shaded}
We observe the rune \kode{?:}, which happens to mean the same thing it
means in C.  In C, though, we don't call it \kode{wutcol}.

(In case you don't remember from chapter 5, \kode{\&} just means \kode{yes}, 
i.e., \kode{@f}0 in our loobean logic.  (\kode{\textbar{}} is 1.)  So this twig just
produces \kode{47}, not that it matters.)

\kode{\%wutcol} as a noun has a problem---it doesn't fit in 32 bits.
Efficiency is important to us, so we disemvowel: \kode{\%wtcl}.  We
can find this defined in \kode{hoon.hoon}:

\begin{framed_shaded}
\begin{Verbatim}[fontsize=\relsize{-2.5},fontseries=b,commandchars=\\\{\}]
[\PYZpc{}wtcl p=twig q=twig r=twig]
\end{Verbatim}
\end{framed_shaded}
So \kode{?:} has three legs: the test, the yes case, and the no
case.  Again, as in C.  But why do we write it in this strange
triangular indentation?

\subsection{Tall forms}

When all is said and done, the programmer is formatting a big
wall of text.  This canvas has a curious but essential property---it is indefinitely tall, but finitely wide.  We strongly
encourage an 80-column standard.

So the programmer's task as a visual designer is to persuade her
code to flow \emph{down}, not \emph{across}.  The usual way to lay out a
tree which does not fit on one line is to indent the subtrees
and enclose them in parens, brackets or braces.  Which might look
like this (\emph{not} Hoon syntax):

\begin{framed_shaded}
\begin{Verbatim}[fontsize=\relsize{-2.5},fontseries=b,commandchars=\\\{\}]
?:  \PYZob{}
  \PYZam{}
  47
  52
\PYZcb{}
\end{Verbatim}
\end{framed_shaded}
Hoon, like other functional languages, has very deep expression
trees.  In this simple, classic syntax model, a functional
language develops huge piles of closing parens at the end of
large blocks, which is manageable but ugly.  Less manageably, as
each subtree is indented to the right, the width of the text
window bounds the depth of the expression tree.

Other languages skip the braces and parse whitespace, using
indentation to express tree depth.  This actually is valid (but
ugly) Hoon:

\begin{framed_shaded}
\begin{Verbatim}[fontsize=\relsize{-2.5},fontseries=b,commandchars=\\\{\}]
?:
  \PYZam{}
  47
  52
\end{Verbatim}
\end{framed_shaded}
This gets rid of the terminator problem, but keeps the width
problem.  And parsing whitespace is horrible.  Whitespace in Hoon
is not significant, though its presence or absence is.  (Note
also that hard TAB characters are \emph{zutiefst verboten}).

Hoon notices a couple of things about this problem.  First, most
Hoon twigs have small constant fanout.  A parser shouldn't need
either significant whitespace or a terminator to figure out how
many twigs follow \kode{?:}---the answer is always 3.

Second, our goal is to descend into a deep tree without losing
right margin.  With the \emph{backstep} pattern

\begin{framed_shaded}
\begin{Verbatim}[fontsize=\relsize{-2.5},fontseries=b,commandchars=\\\{\}]
?:  \PYZam{}
  47
52
\end{Verbatim}
\end{framed_shaded}
or

\begin{framed_shaded}
\begin{Verbatim}[fontsize=\relsize{-2.5},fontseries=b,commandchars=\\\{\}]
=+  a=3
b
\end{Verbatim}
\end{framed_shaded}
we step two spaces backward at each subtwig, till the last one is
at the same indentation as its parent.

This preserves your right margin in one and only one case---where
the bottom twig is the heaviest.  For example, if we write

\begin{framed_shaded}
\begin{Verbatim}[fontsize=\relsize{-2.5},fontseries=b,commandchars=\\\{\}]
?:  \PYZam{}
  47
?:  |
  52
?:  \PYZam{}
  97
=+  35
b
\end{Verbatim}
\end{framed_shaded}
we see a tree that flows neatly down the screen.  It's obviously
much nicer than, say (\emph{not} Hoon syntax):

\begin{framed_shaded}
\begin{Verbatim}[fontsize=\relsize{-2.5},fontseries=b,commandchars=\\\{\}]
?:  \PYZob{}
  \PYZam{}
  47
  ?:  \PYZob{}
    |
    52
    ?:  \PYZob{}
      \PYZam{}
      97
      =+  \PYZob{}
        35
        b
      \PYZcb{}
    \PYZcb{}
  \PYZcb{}
\PYZcb{}
\end{Verbatim}
\end{framed_shaded}
or any similar abortion.  But its downward flow depends on the
coincidence of the bottom twig being the heavy one:

\begin{framed_shaded}
\begin{Verbatim}[fontsize=\relsize{-2.5},fontseries=b,commandchars=\\\{\}]
?:  \PYZam{}
  ?:  |
    52
  ?:  \PYZam{}
    97
  =+  35
  b
47
\end{Verbatim}
\end{framed_shaded}
To handle this, Hoon has a reasonable selection of reverse hoons,
which have the same semantics but inverse order.  For instance,
if \kode{?:} is ``if,'' \kode{?.} (\kode{wutdot}) is ``unless'':

\begin{framed_shaded}
\begin{Verbatim}[fontsize=\relsize{-2.5},fontseries=b,commandchars=\\\{\}]
?.  \PYZam{}
  47
?:  |
  52
?:  \PYZam{}
  97
=+  35
b
\end{Verbatim}
\end{framed_shaded}

\subsection{Wide forms}

Observe that in the tall syntax, there are always at least \emph{two}
spaces (or one newline) between tokens.  Other than this, nothing
requires anything to be tall.  For instance, it is normal and
only slightly aggressive to write:

\begin{framed_shaded}
\begin{Verbatim}[fontsize=\relsize{-2.5},fontseries=b,commandchars=\\\{\}]
?.  \PYZam{}  47
?:  |  52
?:  \PYZam{}  97
=+  35
b
\end{Verbatim}
\end{framed_shaded}
But we could even go so far as:

\begin{framed_shaded}
\begin{Verbatim}[fontsize=\relsize{-2.5},fontseries=b,commandchars=\\\{\}]
?.  \PYZam{}  47  ?:  |  52  ?:  \PYZam{}  97  =+  35  b
\end{Verbatim}
\end{framed_shaded}
Few would find this readable, which is why Hoon also has a \emph{wide}
syntax:

\begin{framed_shaded}
\begin{Verbatim}[fontsize=\relsize{-2.5},fontseries=b,commandchars=\\\{\}]
?.(\PYZam{} 47 ?:(| 52 ?:(\PYZam{} 97 =+(35 b))))
\end{Verbatim}
\end{framed_shaded}
On a single line, the parentheses---while a parser could get away
with skipping them---are needed to actually read the expression.
The hoon attaches directly to the left paren (\kode{pel}), and a
double space or a newline is a syntax error.

The semantics of tall and wide syntax are identical, of course.
The choice is entirely up to the programmer.  Some languages can
be formatted automatically---turning an abstract syntax tree into
a tall, handsome Hoon file is an art form.  We won't say a
program could never do it---but it'd be work.

Wide forms are also nice because our immature and incomplete
command-line shell can't process multi-line input.

\subsection{Irregular forms}

For a very large set of primitives, neither tall nor wide form is
tight enough.  If you go to \kode{++scat} in \kode{hoon.hoon}, you can see
them all, organized by initial character.

This isn't the place to go over the irregular forms directly---we'll introduce them when we talk about individual runes, or
when we run into them and we can't go around.

\section{Semantics}

We're finally ready to write our first Hoon program.  But\ldots{}

\subsection{A new testbed}

Since we're going to have to write multi-line Hoon programs, the
command line is no longer enough.  We'll need another toy
testbed, this one in \kode{urb/waclux-tomwyc/try/bin/toy.hoon}.
Its text should be:

\begin{framed_shaded}
\begin{Verbatim}[fontsize=\relsize{-2.5},fontseries=b,commandchars=\\\{\}]
!:             ::  To write a trivial Hoon program
|=  *          ::
|=  [x=@ \PYZti{}]    ::  For educational purposes only
:\PYZus{}  \PYZti{}  :\PYZus{}  \PYZti{}   ::
:\PYZhy{}  \PYZpc{}\PYZdl{}         ::  Preserve this mysterious boilerplate square
!\PYZgt{}             ::
:::::::::::::::::  Produce a value below
(add 2 x)
\end{Verbatim}
\end{framed_shaded}
Test that it works: 

\begin{framed_shaded}
\begin{Verbatim}[fontsize=\relsize{-2.5},fontseries=b,commandchars=\\\{\}]
\PYZti{}waclux\PYZhy{}tomwyc/try=\PYZgt{} :toy 3
5
\end{Verbatim}
\end{framed_shaded}
Copy it into \kode{try/bin/hec.hoon}, where we'll write our Hoon decrement.

\begin{framed_shaded}
\begin{Verbatim}[fontsize=\relsize{-2.5},fontseries=b,commandchars=\\\{\}]
+ /\PYZti{}waclux\PYZhy{}tomwyc/try/7/bin/hec/hoon
\PYZti{}waclux\PYZhy{}tomwyc/try=\PYZgt{} :hec 3
5
\end{Verbatim}
\end{framed_shaded}
Replace \kode{(add 2 x)} with our decrement twig:

\begin{framed_shaded}
\begin{Verbatim}[fontsize=\relsize{-2.5},fontseries=b,commandchars=\\\{\}]
\PYZpc{}.  x
|=  a=@
=|  b=@
|\PYZhy{}  ?:  =(a +(b))
      b
    \PYZdl{}(b +(b))

\PYZti{}waclux\PYZhy{}tomwyc/try=\PYZgt{} :hec 42
41
\end{Verbatim}
\end{framed_shaded}
It works.  But, why does it work?  What are these twigs, anyway?

\subsection{Twigs}

We have already defined the noble twig---in chapter 5.  Let's
just \emph{reprint} that text---odds are you've forgotten it already.

When we parse a Hoon expression, file, etc., we produce what we
call a \kode{twig}, which (if you know the CS jargon) is an AST.  A
twig is a noun that's converted into a Nock formula, with
the assistance of a type which describes the subject of the
formula:

\begin{framed_shaded}
\begin{Verbatim}[fontsize=\relsize{-2.5},fontseries=b,commandchars=\\\{\}]
[subject\PYZhy{}type twig] =\PYZgt{} formula
\end{Verbatim}
\end{framed_shaded}
But actually this isn't quite right, because Hoon does something
called ``type inference.''  When we have a type that describes the
subject for the formula we're trying to generate, as we generate
that formula we want to also generate a type for the product of
that formula on that subject.  So our compiler computes:

\begin{framed_shaded}
\begin{Verbatim}[fontsize=\relsize{-2.5},fontseries=b,commandchars=\\\{\}]
[subject\PYZhy{}type twig] =\PYZgt{} [product\PYZhy{}type formula]
\end{Verbatim}
\end{framed_shaded}
As long as \kode{subject-type} is a correct description of some
subject, you can take any twig and compile it against
\kode{subject-type}, producing a \kode{formula} such that \kode{*(subject
formula)} is a product correctly described by \kode{product-type}.

Actually, this works well enough that in Hoon there is no direct
syntax for defining or declaring a type.  There is only a syntax
for constructing twigs.  Types are exclusively formed by
inference.

\subsection{Natural and synthetic hoons}

Fortunately, most kinds of hoons are \emph{synthetic} hoons---in a word,
macros.  Synthetic hoons evaluate by reducing to other twigs,
eventually down to direct ones.  Hoon could do without all its
synthetic hoons, though it would be awfully cumbersome.

For example, as we've seen with \kode{?:} and \kode{?.}, when we compile
\kode{wutdot}---\kode{[\%wtdt p q r]}---we turn it into \kode{[\%wtcl p r q]}.
It's all just syntactic sugar.

In fact, at the risk of scaring you further, here is the entire
Hoon type-inference function from \kode{hoon.hoon}.  \kode{++play} is a 
serviceable list of the \emph{natural} hoons---the axioms, as it
were.  Understand all these, and the rest are just\ldots{} macros.

\begin{framed_shaded}
\begin{Verbatim}[fontsize=\relsize{-2.5},fontseries=b,commandchars=\\\{\}]
++  play
  \PYZti{}/  \PYZpc{}play
  =\PYZgt{}  .(vet |)
  |=  gen=twig
  \PYZca{}\PYZhy{}  type
  ?\PYZhy{}  gen
    [\PYZca{} *]      (cell \PYZdl{}(gen p.gen) \PYZdl{}(gen q.gen))
    [\PYZpc{}bcpt *]  \PYZdl{}(gen (\PYZti{}(whip al q.gen) p:(seep \PYZpc{}read p.gen)))
    [\PYZpc{}brcn *]  (core sut \PYZpc{}gold sut [[\PYZpc{}0 0] p.gen])
    [\PYZpc{}cnts *]  =+  lar=(foil (seek \PYZpc{}read p.gen))
               =+  mew=(snub q.gen)
               =+  rag=q.q.lar
               \PYZpc{}\PYZhy{}  fire
               |\PYZhy{}  \PYZca{}\PYZhy{}  (list ,[p=type q=foot])
               ?@  mew
                 rag
               \PYZdl{}(mew t.mew, rag q:(tock p.i.mew \PYZca{}\PYZdl{}(gen q.i.mew) rag))
    [\PYZpc{}dtkt *]  \PYZpc{}noun
    [\PYZpc{}dtls *]  [\PYZpc{}atom \PYZpc{}\PYZdl{}]
    [\PYZpc{}dtzy *]  ?:(=(\PYZpc{}f p.gen) ?\PYZgt{}((lte q.gen 1) bean) [\PYZpc{}atom p.gen])
    [\PYZpc{}dtzz *]  [\PYZpc{}cube q.gen ?:(.?(q.gen) \PYZpc{}noun [\PYZpc{}atom p.gen])]
    [\PYZpc{}dttr *]  \PYZpc{}noun
    [\PYZpc{}dtts *]  bean
    [\PYZpc{}dtwt *]  bean
    [\PYZpc{}ktbr *]  (wrap(sut \PYZdl{}(gen p.gen)) \PYZpc{}iron)
    [\PYZpc{}ktls *]  \PYZdl{}(gen p.gen)
    [\PYZpc{}ktpm *]  (wrap(sut \PYZdl{}(gen p.gen)) \PYZpc{}zinc)
    [\PYZpc{}ktsg *]  \PYZdl{}(gen p.gen)
    [\PYZpc{}ktts *]  (conk(sut \PYZdl{}(gen q.gen)) p.gen)
    [\PYZpc{}ktwt *]  (wrap(sut \PYZdl{}(gen p.gen)) \PYZpc{}lead)
    [\PYZpc{}sgzp *]  \PYZti{}\PYZus{}(duck(sut \PYZca{}\PYZdl{}(gen p.gen)) \PYZdl{}(gen q.gen))
    [\PYZpc{}sggr *]  \PYZdl{}(gen q.gen)
    [\PYZpc{}tsgr *]  \PYZdl{}(gen q.gen, sut \PYZdl{}(gen p.gen))
    [\PYZpc{}tstr *]  \PYZdl{}(gen r.gen, sut (busk p.gen q.gen))
    [\PYZpc{}wtcl *]  =+  [fex=(gain p.gen) wux=(lose p.gen)]
               \PYZpc{}+  fork
                 ?:(=(\PYZpc{}void fex) \PYZpc{}void \PYZdl{}(sut fex, gen q.gen))
               ?:(=(\PYZpc{}void wux) \PYZpc{}void \PYZdl{}(sut wux, gen r.gen))
    [\PYZpc{}zpcb *]  \PYZti{}\PYZus{}((show \PYZpc{}o p.gen) \PYZdl{}(gen q.gen))
    [\PYZpc{}zpcm *]  (play p.gen)
    [\PYZpc{}zpcn \PYZti{}]  p:seed
    [\PYZpc{}zpfs *]  \PYZpc{}void
    [\PYZpc{}zpsm *]  (cell \PYZdl{}(gen p.gen) \PYZdl{}(gen q.gen))
    [\PYZpc{}zpts *]  \PYZpc{}noun
    [\PYZpc{}zpzp \PYZti{}]  \PYZpc{}void
    *          =+  doz=\PYZti{}(open ap gen)
               ?:  =(doz gen)
                 \PYZti{}\PYZus{}  (show [\PYZpc{}c \PYZsq{}hoon\PYZsq{}] [\PYZpc{}q gen])
                 \PYZti{}|(\PYZpc{}play\PYZhy{}open !!)
               \PYZdl{}(gen doz)
  ==
\end{Verbatim}
\end{framed_shaded}
Well, it's a little intimidating.  But not bad for a whole
language, perhaps.

Let's start by working through the hoons we'll need to make 
decrement work.

\subsection{Cores}

To build decrement, we'll need a loop.  To write a loop, we'll
need a core.  This adds another to the kinds of types we know:

\begin{framed_shaded}
\begin{Verbatim}[fontsize=\relsize{-2.5},fontseries=b,commandchars=\\\{\}]
++  type  \PYZdl{}|  ?(\PYZpc{}noun \PYZpc{}void)                            ::
          \PYZdl{}\PYZpc{}  [\PYZpc{}atom p=term]                            ::
              [\PYZpc{}cell p=type q=type]                     ::
              [\PYZpc{}core p=type q=coil]                     ::
              [\PYZpc{}cube p=* q=type]                        ::
              [\PYZpc{}face p=term q=type]                     ::
              [\PYZpc{}fork p=type q=type]                     ::
          ==                                            ::
++  coil  \PYZdl{}:  p=?(\PYZpc{}gold \PYZpc{}iron \PYZpc{}lead \PYZpc{}zinc)              ::
              q=type                                    ::
              r=[p=?(\PYZti{} \PYZca{}) q=(map term foot)]            ::
          ==                                            ::
++  foot  \PYZdl{}\PYZpc{}  [\PYZpc{}ash p=twig]                             ::
              [\PYZpc{}elm p=twig]                             ::
          ==                                            ::
\end{Verbatim}
\end{framed_shaded}
Aha, you say.  I knew there had to be something complicated in
here.  Well, fact is, I'm just a simple country mouse and so are
you, but Hoon is a polymorphic higher-order typed functional
language with genericity and stuff, and you don't get that
without a little bit of urban funk.

But we want to keep it simple for now, so let's imagine it said

\begin{framed_shaded}
\begin{Verbatim}[fontsize=\relsize{-2.5},fontseries=b,commandchars=\\\{\}]
[\PYZpc{}core p=type q=(map term twig)]
\end{Verbatim}
\end{framed_shaded}
As we recall from chapter 4, a core is a \kode{[code data]} cell---\kode{[battery payload]}.  Essentially, an object.  The battery, at
the Nock level, is a tree of formulas, each of whose subject 
is the core itself.

Let's change our test file to produce a core.  The whole file:

\begin{framed_shaded}
\begin{Verbatim}[fontsize=\relsize{-2.5},fontseries=b,commandchars=\\\{\}]
!:             ::  To write a trivial Hoon program
|=  *          ::
|=  [a=@ \PYZti{}]    ::  For educational purposes only
:\PYZus{}  \PYZti{}  :\PYZus{}  \PYZti{}   ::
:\PYZhy{}  \PYZpc{}\PYZdl{}         ::  Preserve this mysterious boilerplate square
!\PYZgt{}             ::
:::::::::::::::::  Produce a value below
|\PYZpc{}
++  hello
  \PYZdq{}hello, world.\PYZdq{}
\PYZhy{}\PYZhy{}
\end{Verbatim}
\end{framed_shaded}

The syntax for a basic core is \kode{\textbar{}\%} (\kode{barcen}, or \kode{\%brcn}),
followed by any number of arms \kode{++} (\kode{luslus}, or just \kode{slus}),
followed by a terminator \phep\ (\kode{hephep}, or just \kode{phep}).  The arm
is a symbol and a twig.  The subject of the twig is the core.

Let's try this puppy out:

\begin{framed_shaded}
\begin{Verbatim}[fontsize=\relsize{-2.5},fontseries=b,commandchars=\\\{\}]
: /\PYZti{}waclux\PYZhy{}tomwyc/try/9/bin/hec/hoon
\PYZti{}waclux\PYZhy{}tomwyc/try=\PYZgt{} :hec
! type\PYZhy{}fail
! exit
\end{Verbatim}
\end{framed_shaded}
Whoops!  We forgot that Arvo is a \emph{typed} command-line.  Because
it's parsed as an open-ended list, it always has the terminator
(\kode{\sig }, which is just \kode{@n}0) on the end.  But it needs an atom:

\begin{framed_shaded}
\begin{Verbatim}[fontsize=\relsize{-2.5},fontseries=b,commandchars=\\\{\}]
|=  [a=@ \PYZti{}]    ::  For educational purposes only
\end{Verbatim}
\end{framed_shaded}
Hence:

\begin{framed_shaded}
\begin{Verbatim}[fontsize=\relsize{-2.5},fontseries=b,commandchars=\\\{\}]
\PYZti{}waclux\PYZhy{}tomwyc/try=\PYZgt{} :hec 42
\PYZlt{} 1.ivl
  1.hfd
  [ [a=@ \PYZpc{}\PYZti{}]
    \PYZlt{}1.vpy [* [@p /] \PYZlt{}218.tvj 18.olk 323.uvl 81.wza 1.xlc \PYZpc{}164\PYZgt{}]\PYZgt{}
  ]
\PYZgt{}
\end{Verbatim}
\end{framed_shaded}

That's a core.  Or it's how we print a core, anyway.  This is
actually a giant noun full of all kinds of formulas, and it would
be kind of lame to dump a megabyte of nock on your screen.  The
print format is:

\begin{framed_shaded}
\begin{Verbatim}[fontsize=\relsize{-2.5},fontseries=b,commandchars=\\\{\}]
\PYZlt{}number\PYZhy{}of\PYZhy{}arms.checksum payload\PYZgt{}
\end{Verbatim}
\end{framed_shaded}
So we wrapped our new \kode{1.ivl} core, with its \kode{++hello} arm,
around the \kode{1.hfd} core (which is the \kode{\textbar{}=  [a=@ \sig ]} thingy),
around a stack of cores ultimately terminating in the giant
kernel with hundreds of arms (like \kode{218.tvj}).

Okay.  But we built a core because we wanted to use it.  So,
let's do that:
\begin{framed_shaded}
\begin{Verbatim}[fontsize=\relsize{-2.5},fontseries=b,commandchars=\\\{\}]
=\PYZgt{}  |\PYZpc{}
    ++  hello
      \PYZdq{}hello, world.\PYZdq{}
    \PYZhy{}\PYZhy{}
hello
\end{Verbatim}
\end{framed_shaded}

What is this \kode{=\textgreater{}}, \kode{tisgar}, \kode{\%tsgr}?  You remember Nock 7.  \kode{=\textgreater{}}
is Nock 7.  \kode{=\textgreater{}(a b)} means ``use a as the subject of b.''  So, we
are resolving the limb \kode{hello} against our \kode{1.ivl} core.

So when we try it:

\begin{framed_shaded}
\begin{Verbatim}[fontsize=\relsize{-2.5},fontseries=b,commandchars=\\\{\}]
: /\PYZti{}waclux\PYZhy{}tomwyc/try/11/bin/hec/hoon
\PYZti{}waclux\PYZhy{}tomwyc/try=\PYZgt{} :hec 42
\PYZdq{}hello, world.\PYZdq{}
\end{Verbatim}
\end{framed_shaded}
Let's observe a couple of things.  First, an arm is not a method
in the OO sense.  You don't see any arguments on \kode{++hello}.
Rather, the arm is a computed expression---a synthetic attribute,
as it were.  (Can we build a method?  We'll get to that.)

Second, when we're searching for a name in a core, we search the
payload if we don't find an arm.  For example:
\begin{framed_shaded}
\begin{Verbatim}[fontsize=\relsize{-2.5},fontseries=b,commandchars=\\\{\}]
=\PYZgt{}  |\PYZpc{}
    ++  hello
      \PYZdq{}hello, world.\PYZdq{}
    \PYZhy{}\PYZhy{}
a

: /\PYZti{}waclux\PYZhy{}tomwyc/try/12/bin/hec/hoon
\PYZti{}waclux\PYZhy{}tomwyc/try=\PYZgt{} :hec 42
42
\end{Verbatim}
\end{framed_shaded}

Our command-line argument, \kode{a}, is still in there.  We can also
use it from within the arm.  With some string-interpolation black
magic:
\begin{framed_shaded}
\begin{Verbatim}[fontsize=\relsize{-2.5},fontseries=b,commandchars=\\\{\}]
=\PYZgt{}  |\PYZpc{}
    ++  hello
      \PYZdq{}hello, world\PYZhy{}\PYZhy{}\PYZhy{}a is \PYZob{} (scow \PYZpc{}ud a) \PYZcb{}.\PYZdq{}
    \PYZhy{}\PYZhy{}
hello

: /\PYZti{}waclux\PYZhy{}tomwyc/try/13/bin/hec/hoon
\PYZti{}waclux\PYZhy{}tomwyc/try=\PYZgt{} :hec 42
\PYZdq{}hello, world\PYZhy{}\PYZhy{}\PYZhy{}a is 42.\PYZdq{}
\end{Verbatim}
\end{framed_shaded}

Finally, this is a classic case in which the twig needs to be
reversed to make it flow downward.  We need the opposite of \kode{=\textgreater{}}:
\kode{=\textless{}}, \kode{tisgal}, \kode{\%tsgl}:
\begin{framed_shaded}
\begin{Verbatim}[fontsize=\relsize{-2.5},fontseries=b,commandchars=\\\{\}]
=\PYZlt{}  hello
|\PYZpc{}
++  hello
  \PYZdq{}hello, world\PYZhy{}\PYZhy{}\PYZhy{}a is \PYZob{} (scow \PYZpc{}ud a) \PYZcb{}.\PYZdq{}
\PYZhy{}\PYZhy{}

: /\PYZti{}waclux\PYZhy{}tomwyc/try/14/bin/hec/hoon
\PYZti{}waclux\PYZhy{}tomwyc/try=\PYZgt{} :hec 42
\PYZdq{}hello, world\PYZhy{}\PYZhy{}\PYZhy{}a is 42.\PYZdq{}
\end{Verbatim}
\end{framed_shaded}

\subsection{Preparing to decrement}

To do some decrementing, we'll need a counter.  Let's continue
our pattern of using only natural hoons:

\begin{framed_shaded}
\begin{Verbatim}[fontsize=\relsize{-2.5},fontseries=b,commandchars=\\\{\}]
=\PYZgt{}  :\PYZhy{}  \PYZca{}=  b
        0
    .
=\PYZlt{}  decrement
|\PYZpc{}
++  decrement
  b
\PYZhy{}\PYZhy{}

: /\PYZti{}waclux\PYZhy{}tomwyc/try/16/bin/hec/hoon
\PYZti{}waclux\PYZhy{}tomwyc/try=\PYZgt{} :hec 42
0
\end{Verbatim}
\end{framed_shaded}

We introduce a couple of new hoons.  First, \kode{\ket =}, \kode{kettis},
\kode{\%ktts}---is a hoon we've already seen.  We've seen it only in
its irregular form---not \kode{\ket =(b 0)}, but, of course, \kode{b=0}.
(Pronounced not ``b tis zero,'' but, of course, ``b is zero.'')

We've also seen \kode{:-} in its irregular form---it just makes a
cell.  \kode{:-(a b)} is just \kode{[a b]}.  We have a set of these hoons,
which let us build cells in classic Hoon fashion:

\begin{framed_shaded}
\begin{Verbatim}[fontsize=\relsize{-2.5},fontseries=b,commandchars=\\\{\}]
:\PYZhy{}(a b)       [a b]
:+(a b c)     [a b c]
:\PYZca{}(a b c d)   [a b c d]
:\PYZus{}(a b)       [b a]
\end{Verbatim}
\end{framed_shaded}
So we might as well say
\begin{framed_shaded}
\begin{Verbatim}[fontsize=\relsize{-2.5},fontseries=b,commandchars=\\\{\}]
=\PYZgt{}  [b=0 .]
=\PYZlt{}  decrement
|\PYZpc{}
++  decrement
  b
\PYZhy{}\PYZhy{}
\end{Verbatim}
\end{framed_shaded}

In other words---enter our core not with the original subject,
\kode{.}, but with the cell \kode{[b=0 .]}.  

You might remember this as nock \kode{8}---and in fact (it is a
synthetic hoon, but the compiler sees what you're doing and turns
it into nock \kode{8} anyway) there's a hoon for that: \kode{=+}, \kode{tislus},
\kode{\%tsls}:
\begin{framed_shaded}
\begin{Verbatim}[fontsize=\relsize{-2.5},fontseries=b,commandchars=\\\{\}]
=+  b=0
=\PYZlt{}  decrement
|\PYZpc{}
++  decrement
  b
\PYZhy{}\PYZhy{}
\end{Verbatim}
\end{framed_shaded}

\subsection{Actually decrementing}

As you may remember, to decrement \kode{a} we need to count up to it.
The first step is incrementing, which we do with \kode{.+}---\kode{dotlus}, 
\kode{\%dtls}:
\begin{framed_shaded}
\begin{Verbatim}[fontsize=\relsize{-2.5},fontseries=b,commandchars=\\\{\}]
=+  b=0
=\PYZlt{}  decrement
|\PYZpc{}
++  decrement
  .+(b)
\PYZhy{}\PYZhy{}

: /\PYZti{}waclux\PYZhy{}tomwyc/try/19/bin/hec/hoon
\PYZti{}waclux\PYZhy{}tomwyc/try=\PYZgt{} :hec 42
1
\PYZti{}waclux\PYZhy{}tomwyc/try=\PYZgt{} 
\end{Verbatim}
\end{framed_shaded}

Well, it's not decrement but it's a start.  What we really have
to do is see if \kode{+(b)} equals \kode{a}:
\begin{framed_shaded}
\begin{Verbatim}[fontsize=\relsize{-2.5},fontseries=b,commandchars=\\\{\}]
=+  b=0
=\PYZlt{}  decrement
|\PYZpc{}
++  decrement
  ?:  .=(a +(b))
    b
  99
\PYZhy{}\PYZhy{}
\end{Verbatim}
\end{framed_shaded}

We've met a new hoon, \kode{.=}, \kode{dottis}, \kode{\%dtts}.  It too has an
irregular form, not surprisingly different:
\begin{framed_shaded}
\begin{Verbatim}[fontsize=\relsize{-2.5},fontseries=b,commandchars=\\\{\}]
=+  b=0
=\PYZlt{}  decrement
|\PYZpc{}
++  decrement
  ?:  =(a +(b))
    b
  99
\PYZhy{}\PYZhy{}
\end{Verbatim}
\end{framed_shaded}

And \kode{?:} was of course one of our first examples.  We can test
this, for what it's worth:

\begin{framed_shaded}
\begin{Verbatim}[fontsize=\relsize{-2.5},fontseries=b,commandchars=\\\{\}]
\PYZti{}waclux\PYZhy{}tomwyc/try=\PYZgt{} :hec 42
99
\PYZti{}waclux\PYZhy{}tomwyc/try=\PYZgt{} :hec 1
0
\end{Verbatim}
\end{framed_shaded}
Well, it works.  For one case of ``works.''

You know, gosh, instead of \kode{99}---which is obviously just wrong---what we'd actually like to do is, if \kode{+(b)} isn't equal to \kode{a},
compute \kode{decrement} again, but with \kode{b} set to \kode{+(b)}.

There's a way to do that:
\begin{framed_shaded}
\begin{Verbatim}[fontsize=\relsize{-2.5},fontseries=b,commandchars=\\\{\}]
=+  b=0
=\PYZlt{}  decrement
|\PYZpc{}
++  decrement
  ?:  =(a +(b))
    b
  \PYZpc{}=  decrement
      b  +(b)
  ==
\PYZhy{}\PYZhy{}
\end{Verbatim}
\end{framed_shaded}

Meet \kode{\%=}, \kode{centis}, \kode{\%cnts}---the world's most important hoon.
Actually, \emph{everything} that references a limb/wing turns into
\kode{\%=}.  Let's look at its definition:

\begin{framed_shaded}
\begin{Verbatim}[fontsize=\relsize{-2.5},fontseries=b,commandchars=\\\{\}]
++  twig  \PYZdl{}\PYZpc{}  [\PYZpc{}cnts p=wing q=tray]
          ==
++  tray  (list ,[p=wing q=twig])
\end{Verbatim}
\end{framed_shaded}
\kode{\%=} means ``resolve with changes.''  If \kode{q} is empty, \kode{\%=} just 
pulls wing \kode{p} with no changes.  Otherwise, we get \kode{p} with the
wings in \kode{q} set to the provided twigs.

A wing, of course, can resolve to a leg or an arm---a fragment of
the subject, or a computed attribute like \kode{++decrement} above.
When \kode{p} resolves to an arm, we compute based on the changes
defined in \kode{q}.  (When one of the wings in \kode{q} resolves to an
arm, the change is to the core that contains the arm.)

So, this should do exactly what we want---it should replace \kode{b}
with \kode{+(b)}, and recompute.  But does it?  Amazingly\ldots{}

\begin{framed_shaded}
\begin{Verbatim}[fontsize=\relsize{-2.5},fontseries=b,commandchars=\\\{\}]
: /\PYZti{}waclux\PYZhy{}tomwyc/try/21/bin/hec/hoon
\PYZti{}waclux\PYZhy{}tomwyc/try=\PYZgt{} :hec 42
41
\end{Verbatim}
\end{framed_shaded}

\subsection{Making it pretty}

The first thing we notice is that \kode{\%=} is pretty important, and
being pretty important it ought to have an irregular form:
\begin{framed_shaded}
\begin{Verbatim}[fontsize=\relsize{-2.5},fontseries=b,commandchars=\\\{\}]
=+  b=0
=\PYZlt{}  decrement
|\PYZpc{}
++  decrement
  ?:  =(a +(b))
    b
  decrement(b +(b))
\PYZhy{}\PYZhy{}
\end{Verbatim}
\end{framed_shaded}

The second thing we notice is this heavy word, \kode{decrement}, which
we are dragging around everywhere.  Actually, we know we're
writing a decrement program.  So why do we keep saying decrement,
decrement, decrement?

Naming things is one of the most annoying and difficult problems
in programming.  Nobody should have to name anything, especially
if its only job is to call itself.

Fortunately, a unique feature of Hoon is \emph{the empty name}, \kode{\$}.
The character (the dollar sign) is pronounced \kode{buc}; the value;
(the empty term) is pronounced \kode{blip}.  So you can say either.
Or both---depends if you're money.  Either way, neighbor, \kode{\$}
has a value and it ain't nothing but zero:

\begin{framed_shaded}
\begin{Verbatim}[fontsize=\relsize{-2.5},fontseries=b,commandchars=\\\{\}]
\PYZti{}waclux\PYZhy{}tomwyc/try=\PYZgt{} `@ux`\PYZpc{}foo
0x6f.6f66
\PYZti{}waclux\PYZhy{}tomwyc/try=\PYZgt{} `@ux`\PYZpc{}\PYZdl{}
0x0
\end{Verbatim}
\end{framed_shaded}
When we use \kode{\$} as a name, our decrement gets cleaner---or
shorter, anyway:
\begin{framed_shaded}
\begin{Verbatim}[fontsize=\relsize{-2.5},fontseries=b,commandchars=\\\{\}]
=+  b=0
=\PYZlt{}  \PYZdl{}
|\PYZpc{}
++  \PYZdl{}
  ?:  =(a +(b))
    b
  \PYZdl{}(b +(b))
\PYZhy{}\PYZhy{}
\end{Verbatim}
\end{framed_shaded}

The third thing we notice is that this pattern of ``have one arm
and do it again with some changes'' is\ldots{} well, it has a name.
So we might expect to see a more convenient hoon---and indeed
we do:

\begin{framed_shaded}
\begin{Verbatim}[fontsize=\relsize{-2.5},fontseries=b,commandchars=\\\{\}]
=+  b=0
=\PYZlt{}  \PYZdl{}
|.
?:  =(a +(b))
  b
\PYZdl{}(b +(b))
\end{Verbatim}
\end{framed_shaded}
What is \kode{\textbar{}.}, \kode{bardot}, \kode{\%brdt}?  It's easy to see what a
synthetic hoon does---we just look at its line in \kode{++open}
(in \kode{hoon.hoon}).  For example:

\begin{framed_shaded}
\begin{Verbatim}[fontsize=\relsize{-2.5},fontseries=b,commandchars=\\\{\}]
[\PYZpc{}tsgl *]  [\PYZpc{}tsgr q.gen p.gen]
\end{Verbatim}
\end{framed_shaded}
and, more to the point:

\begin{framed_shaded}
\begin{Verbatim}[fontsize=\relsize{-2.5},fontseries=b,commandchars=\\\{\}]
[\PYZpc{}brdt *]  [\PYZpc{}brcn (\PYZti{}(put by *(map term foot)) \PYZpc{}\PYZdl{} [\PYZpc{}ash p.gen])]
\end{Verbatim}
\end{framed_shaded}
(Okay, this isn't exactly as pellucid as it might be.  All it
means is that we create a \kode{\textbar{}\%} with one arm, our \kode{\%\$}.)

Moreover, it seems like we might want to activate one of these
strange repeating computations automagically.  Indeed there's a
hoon for that---\kode{\textbar{}-}, \kode{barhep}, \kode{\%brhp}:

\begin{framed_shaded}
\begin{Verbatim}[fontsize=\relsize{-2.5},fontseries=b,commandchars=\\\{\}]
[\PYZpc{}brhp *]  [\PYZpc{}tsgl [\PYZpc{}cnzy \PYZpc{}\PYZdl{}] [\PYZpc{}brdt p.gen]]
\end{Verbatim}
\end{framed_shaded}
Notice that we just did that.  \kode{\%cnzy} is an internal hoon which
doesn't have a syntax, and just makes macros smaller:

\begin{framed_shaded}
\begin{Verbatim}[fontsize=\relsize{-2.5},fontseries=b,commandchars=\\\{\}]
[\PYZpc{}cnzy *]  [\PYZpc{}cnts [p.gen \PYZti{}] \PYZti{}]
\end{Verbatim}
\end{framed_shaded}
Aha, good old \kode{\%cnts}---aka, \kode{\%=}.  But wait---does it work?
We get:

\begin{framed_shaded}
\begin{Verbatim}[fontsize=\relsize{-2.5},fontseries=b,commandchars=\\\{\}]
=+  b=0
|\PYZhy{}
?:  =(a +(b))
  b
\PYZdl{}(b +(b))
\end{Verbatim}
\end{framed_shaded}
Or, to indent this a little more aggressively, our final result.
It isn't exactly what we started with---but we'll get there in 
a minute:

\begin{framed_shaded}
\begin{Verbatim}[fontsize=\relsize{-2.5},fontseries=b,commandchars=\\\{\}]
!:             ::  To write a trivial Hoon program
|=  *          ::
|=  [a=@ \PYZti{}]    ::  For educational purposes only
:\PYZus{}  \PYZti{}  :\PYZus{}  \PYZti{}   ::
:\PYZhy{}  \PYZpc{}\PYZdl{}         ::  Preserve this mysterious boilerplate square
!\PYZgt{}             ::
:::::::::::::::::  Produce a value below
=+  b=0
|\PYZhy{}  ?:  =(a +(b))
      b
    \PYZdl{}(b +(b))
\end{Verbatim}
\end{framed_shaded}
Let's try it\ldots{}

\begin{framed_shaded}
\begin{Verbatim}[fontsize=\relsize{-2.5},fontseries=b,commandchars=\\\{\}]
: /\PYZti{}waclux\PYZhy{}tomwyc/try/22/bin/hec/hoon
\PYZti{}waclux\PYZhy{}tomwyc/try=\PYZgt{} :hec 42
41
\end{Verbatim}
\end{framed_shaded}
It works!

\section{Exercises}

Exercise 1: write the entire expression under the boilerplate
above, in one line as wide form.  Test it.

Exercise 2: write a Hoon program :fib that instead computes the
Fibonacci number at \kode{n}.