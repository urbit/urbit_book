\chapter{Setup}

\begin{quote}
\noindent \emph{Tlön será un laberinto, pero es un laberinto urdido por hombres, un
laberinto destinado a que lo descifren los hombres.}
\medskip \newline
\noindent \emph{Tlön is surely a labyrinth, but it is a labyrinth devised
by men, a labyrinth destined to be deciphered by men.}
\medskip \newline
\noindent ---\href{https://en.wikipedia.org/wiki/Tlon,_Uqbar,_Orbis_Tertius}{\textbf{\emph{Tlön, Uqbar, Orbis Tertius}}}, Jorge Luis Borges
\end{quote}

\section{Prepare your computer}

Urbit runs on Unix machines only.  It depends on:

\begin{itemize}
\item gmp
\item libsigsegv
\item openssl
\item libssl-dev (Linux only)
\item ncurses (Linux only)
\end{itemize}

Currently we support OSX, Linux (not all distributions have been
tested) and *BSD.  There are no instructions for BSD, because
only people with a serious clue run BSD.  Intrepid ninjas may
attempt ports to other OSes.  If you're not an intrepid ninja,
try a VM (eg, VirtualBox).

\subsection{Configure OS X}

\begin{enumerate}
\item Do you have XCode?  Type \kode{gcc}.  If it says \kode{no input files}, you have XCode.

Otherwise, install XCode: \kode{https://developer.apple.com/xcode/}, with the
command line tools.
\item Install dependencies

\begin{itemize}
\item Do you have Homebrew?  Type \kode{brew}.  If it does something, you have Homebrew.

Otherwise, \kode{ruby -e "\$(curl -fsSL https://raw.github.com/mxcl/homebrew/go)"}
will install it.

And follow up with \kode{sudo brew install gmp libsigsegv openssl}

This will ask you for the root password, which ideally you know.
\item Macports? Type \kode{port}.  If it does something, you have Macports.

Otherwise go \href{http://www.macports.org/install.php}{here}.

Then \kode{sudo port install gmp libsigsegv openssl}

Enter your root password at the prompt.
\end{itemize}
\end{enumerate}

\subsection{Configure Linux (Ubuntu or Debian)}

\begin{enumerate}
\item \kode{sudo apt-get install libgmp3-dev libsigsegv-dev openssl libssl-dev libncurses5-dev git make exuberant-ctags}
\end{enumerate}

\subsection{Configure Linux (AWS)}

\begin{enumerate}
\item \kode{sudo yum --enablerepo epel install gcc git gmp-devel openssl-devel ncurses-devel libsigsegv-devel ctags}
\end{enumerate}

\subsection{Get the source}

Either:

A. Download and unzip \kode{https://github.com/urbit/urbit/archive/master.zip}.

B. \kode{git clone https://github.com/urbit/urbit.git}.

\subsection{Set up your enviroment}

\kode{cd} to the unpacked Urbit directory you just created.  If this works,
\kode{ls urb} should show:

\begin{framed_shaded}
\begin{Verbatim}[fontsize=\relsize{-2.5},fontseries=b,commandchars=\\\{\}]
urbit.pill  zod/
\end{Verbatim}
\end{framed_shaded}

Great!  Now, let's do some dirty Unix stuff to set up your environment.
If you know what this is doing, feel free to do it right.  Otherwise:

\begin{framed_shaded}
\begin{Verbatim}[fontsize=\relsize{-2.5},fontseries=b,commandchars=\\\{\}]
echo \PYZdq{}export URBIT\PYZus{}HOME=`pwd`/urb\PYZdq{} \PYZgt{}\PYZgt{}\PYZti{}/.bash\PYZus{}profile
source \PYZti{}/.bash\PYZus{}profile
\end{Verbatim}
\end{framed_shaded}

To make sure this worked,

\begin{framed_shaded}
\begin{Verbatim}[fontsize=\relsize{-2.5},fontseries=b,commandchars=\\\{\}]
echo \PYZdl{}URBIT\PYZus{}HOME
\end{Verbatim}
\end{framed_shaded}

should show \kode{/urb} within the current directory.

\subsection{Build}

\kode{make}.  Sometimes things are just easy.

\subsection{Run}

Run \kode{bin/vere -c mypier}, where \kode{mypier} is a directory that doesn't yet exist.
All your state (an append-only log and a memory checkpoint) will live in this
directory.  Its name doesn't matter and is not visible internally.

A \emph{pier} is an Urbit virtual machine that hosts one or more Urbit identities,
or \emph{ships}.  When you run \kode{vere -c}, it automatically creates a 128-bit ship,
or \kode{submarine}.  Your name (a hash of a randomly-generated public key) will
look like:

\begin{framed_shaded}
\begin{Verbatim}[fontsize=\relsize{-2.5},fontseries=b,commandchars=\\\{\}]
\PYZti{}machec\PYZhy{}binnev\PYZhy{}dordeb\PYZhy{}sogduc\PYZhy{}\PYZhy{}dosmul\PYZhy{}sarrum\PYZhy{}faplec\PYZhy{}nidted
\end{Verbatim}
\end{framed_shaded}

First you'll see a string of messages like:

\begin{framed_shaded}
\begin{Verbatim}[fontsize=\relsize{-2.5},fontseries=b,commandchars=\\\{\}]
vere: urbit home is /Users/cyarvin/Documents/src/u3/urb
loom: mapped 1024MB
time: \PYZti{}2013.9.1..03.57.11..4935
ames: on localhost, UDP 63908.
generating 2048\PYZhy{}bit RSA pair...
\end{Verbatim}
\end{framed_shaded}

and then it'll pause a little, `cause this is slow\ldots{} and then

\begin{framed_shaded}
\begin{Verbatim}[fontsize=\relsize{-2.5},fontseries=b,commandchars=\\\{\}]
saving passcode in /Users/cyarvin/.urbit/\PYZti{}magsut\PYZhy{}hopful.txt
(for real security, write it down and delete the file...)
\end{Verbatim}
\end{framed_shaded}

and, then, if the network gods are happy, your submarine will start pulling
down Arvo files:

\begin{framed_shaded}
\begin{Verbatim}[fontsize=\relsize{-2.5},fontseries=b,commandchars=\\\{\}]
/\PYZti{}machec\PYZhy{}binnev\PYZhy{}dordeb\PYZhy{}sogduc\PYZhy{}\PYZhy{}dosmul\PYZhy{}sarrum\PYZhy{}faplec\PYZhy{}dted/main/1/bin/ticket/hoon
/\PYZti{}machec\PYZhy{}binnev\PYZhy{}dordeb\PYZhy{}sogduc\PYZhy{}\PYZhy{}dosmul\PYZhy{}sarrum\PYZhy{}faplec\PYZhy{}dted/main/1/bin/reset/hoon
/\PYZti{}machec\PYZhy{}binnev\PYZhy{}dordeb\PYZhy{}sogduc\PYZhy{}\PYZhy{}dosmul\PYZhy{}sarrum\PYZhy{}faplec\PYZhy{}nidted/main/1/bin/ye/hoon
/\PYZti{}machec\PYZhy{}binnev\PYZhy{}dordeb\PYZhy{}sogduc\PYZhy{}\PYZhy{}dosmul\PYZhy{}sarrum\PYZhy{}faplec\PYZhy{}nidted/main/1/bin/ls/hoon
\end{Verbatim}
\end{framed_shaded}

and the like.  You'll see a couple pages of this stuff.  Don't worry too much
about the details right now.  Finally, you'll get the Arvo shell prompt (which
is also a Hoon REPL):

\begin{framed_shaded}
\begin{Verbatim}[fontsize=\relsize{-2.5},fontseries=b,commandchars=\\\{\}]
\PYZti{}machec\PYZhy{}binnev\PYZhy{}dordeb\PYZhy{}sogduc\PYZhy{}\PYZhy{}dosmul\PYZhy{}sarrum\PYZhy{}faplec\PYZhy{}nidted/try=\PYZgt{}
\end{Verbatim}
\end{framed_shaded}

\subsection{Register}

Next, you need to decide whether a mere submarine is enough for
you right now.  This monicker is a mouthful.  You can stick with
it (for now), but\ldots{} you're going to need a wider xterm.

Which might be fine!  However, please note that just by sending a
simple email, you can get a much better ship---a \kode{destroyer},
with a nice short name like

\begin{framed_shaded}
\begin{Verbatim}[fontsize=\relsize{-2.5},fontseries=b,commandchars=\\\{\}]
\PYZti{}waclux\PYZhy{}tomwyc
\end{Verbatim}
\end{framed_shaded}

Just email \kode{urbit@urbit.org}, with your submarine in the subject.
We'll send you destroyers---not one, but \emph{two}.  Yes, two!  Tell
us something cool in the body, and we'll send you even more.

If you have a destroyer, you need to configure it.  Otherwise,
just stretch that xterm wide and skip to section 1.2.

Your destroyers will arrive in the form of [ship ticket] pairs.
Let's say one of your ships is \kode{\sig waclux-tomwyc} and its ticket is

\begin{framed_shaded}
\begin{Verbatim}[fontsize=\relsize{-2.5},fontseries=b,commandchars=\\\{\}]
\PYZti{}ribdyr\PYZhy{}famtem\PYZhy{}larrun\PYZhy{}figtyd
\end{Verbatim}
\end{framed_shaded}

(What are these strings, anyway?  Just random unsigned integers,
rendered in Hoon's syllabic base, \kode{@p}.)

A new life awaits you on the off-world colonies!  To begin, just
type at the prompt:

\begin{framed_shaded}
\begin{Verbatim}[fontsize=\relsize{-2.5},fontseries=b,commandchars=\\\{\}]
:begin \PYZti{}waclux\PYZhy{}tomwyc
\end{Verbatim}
\end{framed_shaded}

and follow the directions.  When the script completes, hit return
and you'll be the \kode{\sig waclux-tomwyc} you wanted to be.

\section{Play with Arvo}

If all went well, you now have a nice short prompt:

\begin{framed_shaded}
\begin{Verbatim}[fontsize=\relsize{-2.5},fontseries=b,commandchars=\\\{\}]
\PYZti{}waclux\PYZhy{}tomwyc/try=\PYZgt{}
\end{Verbatim}
\end{framed_shaded}

If all did not go well (send us another email), or you're just
too impatient to wait for your destroyer, you have a big long
prompt.  Which is fine, really, just ugly---and all these
exercises will still work.

\subsection{Example commands}

Let's try a few quick things to stretch your fingers.  Type these
command lines and you should see the matching results:

\begin{framed_shaded}
\begin{Verbatim}[fontsize=\relsize{-2.5},fontseries=b,commandchars=\\\{\}]
\PYZti{}waclux\PYZhy{}tomwyc/try=\PYZgt{} \PYZdq{}hello, world\PYZdq{}
\PYZdq{}hello, world\PYZdq{}

\PYZti{}waclux\PYZhy{}tomwyc/try=\PYZgt{} (add 2 2)
4

\PYZti{}waclux\PYZhy{}tomwyc/try=\PYZgt{} :hello \PYZpc{}world
\PYZdq{}hello, world.\PYZdq{}

\PYZti{}waclux\PYZhy{}tomwyc/try=\PYZgt{} :cat /=main=/bin/hello/hoon
::
::  /=main=/bin/hello/hoon
::
|=  *
|=  [planet=@ta \PYZti{}]
\PYZca{}\PYZhy{}  bowl
:\PYZus{}  \PYZti{}  :\PYZus{}  \PYZti{}
:\PYZhy{}  \PYZpc{}\PYZpc{}
!\PYZgt{}(\PYZdq{}hello, \PYZob{}(trip planet)\PYZcb{}.\PYZdq{})
\end{Verbatim}
\end{framed_shaded}

What did you just do?

One, you used Arvo as a Hoon REPL to print the constant {\tt "hello,
world"}, which is a fancy way to write the Nock noun

\begin{framed_shaded}
\begin{Verbatim}[fontsize=\relsize{-2.5},fontseries=b,commandchars=\\\{\}]
[104 101 108 108 111 44 32 119 111 114 108 100 0]
\end{Verbatim}
\end{framed_shaded}

Two, you called the Hoon \kode{add} function to see that two plus two
is four.  Math seems to work the same on the off-world colonies.

Three, you ran the Arvo application \kode{:hello} with the argument
\kode{\%world}, which is just a fancy way to write the atom
\kode{431.316.168.567} (or, for non-Germans, \kode{431,316,168,567}).  You
might recognize it better as \kode{0x64.6c72.6f77}---the ASCII
characters in LSB first order.

(Is Urbit German?  Sadly, no.  But all our noun print formats are
URL-safe, which dot is and comma isn't.)

And you (4) used the Arvo application :cat to print the Hoon file

\begin{framed_shaded}
\begin{Verbatim}[fontsize=\relsize{-2.5},fontseries=b,commandchars=\\\{\}]
/=main=/bin/hello/hoon
\end{Verbatim}
\end{framed_shaded}

which, supposing your current date is

\begin{framed_shaded}
\begin{Verbatim}[fontsize=\relsize{-2.5},fontseries=b,commandchars=\\\{\}]
\PYZti{}2013.9.1..04.38.31..f259
\end{Verbatim}
\end{framed_shaded}

(i.e., September 1, 2013 at 4:38:31 GMT/LS25 plus 0xf259/65536
seconds), is equivalent to the global path

\begin{framed_shaded}
\begin{Verbatim}[fontsize=\relsize{-2.5},fontseries=b,commandchars=\\\{\}]
/\PYZti{}waclux\PYZhy{}tomwyc/main/\PYZti{}2013.8.23..04.38.31..f259/bin/hello/hoon
\end{Verbatim}
\end{framed_shaded}

which anyone in Urbit can, see and even use---but we're getting
ahead of ourselves.

\subsection{Control characters}

In any case, what we've seen is that Arvo is a dangerous and
powerful operating system which if handled improperly can cause
serious injury or loss of life.  We exaggerate.  Slightly.

The first thing you need to know is how to control this tool.
Try your arrow keys---you'll see that Arvo has traditional Unix
history editing.  Up and down, left and right work, as do the
simple emacs controls:

\begin{framed_shaded}
\begin{Verbatim}[fontsize=\relsize{-2.5},fontseries=b,commandchars=\\\{\}]
\PYZca{}A  go to beginning of line
\PYZca{}B  left arrow
\PYZca{}D  delete next character
\PYZca{}E  go to end of line
\PYZca{}F  right arrow
\PYZca{}K  kill to end of line
\PYZca{}L  clear the screen
\PYZca{}R  search through history
\PYZca{}U  kill the whole line
\PYZca{}Y  yank (restore from kill ring)
\end{Verbatim}
\end{framed_shaded}

Don't expect any other emacs (or even readline---this is not readline, it's
internal to Arvo) commands to work.

There are also some special control keys specific to Arvo.  It's
a good idea to learn these first so that you feel in, um,
control.

First, we'll quit out of an infinite loop with \kode{\^{}C}:

\begin{framed_shaded}
\begin{Verbatim}[fontsize=\relsize{-2.5},fontseries=b,commandchars=\\\{\}]
\PYZti{}waclux\PYZhy{}tomwyc/try=\PYZgt{} :infinite
\end{Verbatim}
\end{framed_shaded}

When you hit return at the end of this line, Arvo will appear to
hang.  Do not be alarmed!  This is not a bug---it means that
we've started running our infinite loop before printing the next
console prompt.  Simply hit \kode{\^{}C}, and you'll see

\begin{framed_shaded}
\begin{Verbatim}[fontsize=\relsize{-2.5},fontseries=b,commandchars=\\\{\}]
! intr
\PYZti{}waclux\PYZhy{}tomwyc/try=\PYZgt{} :infinite
\end{Verbatim}
\end{framed_shaded}

(There may be some stacktrace stuff before the \kode{! intr}, depending
on whether your kernel was compiled with debugging.)

Hit \kode{\^{}U} to delete the line and escape from infinity.  Arvo is a
deterministic OS; you interrupted it while processing an event
that would never terminate.  It returns to the state it was in
before you hit return---as if nothing had ever happened.

You're probably used to using nondeterministic, preemptive OSes,
in which the difference between a waiting task and an
executing event isn't apparent to the user.  Since Arvo is not
preemptive, it has two very different states: waiting and
working.

When Arvo is working, \kode{\^{}C} cancels the event it's working on.
This event never happened.  Don't worry, nothing bad will happen
to your computer.

When Arvo is waiting, use \kode{\^{}D} to end the current task, which is
the task that's currently prompting you.  If there is a live
prompt and the cursor is not at the end, \kode{\^{}D} will delete the
current character---as in Unix.

Try this by running

\begin{framed_shaded}
\begin{Verbatim}[fontsize=\relsize{-2.5},fontseries=b,commandchars=\\\{\}]
\PYZti{}waclux\PYZhy{}tomwyc/try=\PYZgt{} :begin

Do you have a ship and a ticket? yes
\end{Verbatim}
\end{framed_shaded}

Then hit \kode{\^{}D} and you'll be back to the command prompt (which,
unlike in Unix, is not a task itself, but part of the OS).

We don't always want to kill the prompting task.  We often want
to switch between tasks, or between tasks and the command line.
Sort of like switching between windows, except in a command line.
We do this with \kode{\^{}X}.  Try

\begin{framed_shaded}
\begin{Verbatim}[fontsize=\relsize{-2.5},fontseries=b,commandchars=\\\{\}]
\PYZti{}waclux\PYZhy{}tomwyc/try=\PYZgt{} :begin

Do you have a ship and a ticket? yes
\end{Verbatim}
\end{framed_shaded}

But hit \kode{\^{}X} instead of \kode{\^{}D}.  You'll get a prompt again.  Use
it:

\begin{framed_shaded}
\begin{Verbatim}[fontsize=\relsize{-2.5},fontseries=b,commandchars=\\\{\}]
\PYZti{}waclux\PYZhy{}tomwyc/try=\PYZgt{} :begin

\PYZti{}waclux\PYZhy{}tomwyc/try=\PYZgt{} :hello \PYZpc{}world
\PYZdq{}hello, world.\PYZdq{}
\PYZti{}waclux\PYZhy{}tomwyc/try=\PYZgt{}
\end{Verbatim}
\end{framed_shaded}

Hit \kode{\^{}X} again:

\begin{framed_shaded}
\begin{Verbatim}[fontsize=\relsize{-2.5},fontseries=b,commandchars=\\\{\}]
\PYZti{}waclux\PYZhy{}tomwyc/try=\PYZgt{} :begin

\PYZti{}waclux\PYZhy{}tomwyc/try=\PYZgt{} :hello \PYZpc{}world
\PYZdq{}hello, world.\PYZdq{}
Do you have a ship and a ticket? yes
\end{Verbatim}
\end{framed_shaded}

And finally, hit \kode{\^{}C} to kill the task.

There's one more magic control key that switches your whole
reality.  This is \kode{\^{}W}, which switches between the ships in a pier.
Do you have multiple ships in your pier?  Sure---you still have
your old submarine.  Hit \kode{\^{}W}:

\begin{framed_shaded}
\begin{Verbatim}[fontsize=\relsize{-2.5},fontseries=b,commandchars=\\\{\}]
\PYZti{}machec\PYZhy{}binnev\PYZhy{}dordeb\PYZhy{}sogduc\PYZhy{}\PYZhy{}dosmul\PYZhy{}sarrum\PYZhy{}faplec\PYZhy{}nidted/try=\PYZgt{}
\end{Verbatim}
\end{framed_shaded}

Hit \kode{\^{}W} again:

\begin{framed_shaded}
\begin{Verbatim}[fontsize=\relsize{-2.5},fontseries=b,commandchars=\\\{\}]
\PYZti{}waclux\PYZhy{}tomwyc/try=\PYZgt{}
\end{Verbatim}
\end{framed_shaded}

Finally, Arvo is a single-level store.  Since it's not the '70s
anymore and disk is cheap, everything you do is saved for ever.
(In fact, it's saved in two ways---as a memory image and an event
log---so you, or the government if they haz your filez, can
repeat every computation you've every performed.)

If the current prompt is just the shell prompt, \kode{\^{}D} on an empty
line will log out---as in Unix:

\begin{framed_shaded}
\begin{Verbatim}[fontsize=\relsize{-2.5},fontseries=b,commandchars=\\\{\}]
\PYZti{}waclux\PYZhy{}tomwyc/try=\PYZgt{}
oxford:\PYZti{}/urbit; pwd
/Users/cyarvin/urbit
oxford:\PYZti{}/urbit; echo \PYZdq{}hello, world\PYZdq{}
hello, world
oxford:\PYZti{}/urbit;
\end{Verbatim}
\end{framed_shaded}

Then you can restart and be right back where you were---just
run \kode{vere} without \kode{-c}:

\begin{framed_shaded}
\begin{Verbatim}[fontsize=\relsize{-2.5},fontseries=b,commandchars=\\\{\}]
oxford:\PYZti{}/urbit; bin/vere mypier
vere: urbit home is /Users/cyarvin/urb
loom: loaded 9MB
time: \PYZti{}2013.9.1..17.23.05..0cc1
ames: on localhost, UDP 60342.
http: live on 8080
rest: checkpoint to event 383
rest: old 0v1c.gkr1o, new 0v10.m4gdu
\PYZhy{}\PYZhy{}\PYZhy{}\PYZhy{}\PYZhy{}\PYZhy{}\PYZhy{}\PYZhy{}\PYZhy{}\PYZhy{}\PYZhy{}\PYZhy{}\PYZhy{}\PYZhy{}\PYZhy{}\PYZhy{} playback complete\PYZhy{}\PYZhy{}\PYZhy{}\PYZhy{}\PYZhy{}\PYZhy{}\PYZhy{}\PYZhy{}\PYZhy{}\PYZhy{}\PYZhy{}\PYZhy{}\PYZhy{}\PYZhy{}\PYZhy{}\PYZhy{}
waclux\PYZhy{}tomwyc/try=\PYZgt{}
\end{Verbatim}
\end{framed_shaded}

Use your arrow keys and you'll see your history is still there.
Arvo is indestructible and can be shut down however you like
without losing data.  Also, starting a new task while an old
one is still running will kill the old one safely.

But don't try to operate the same ship on two Unix hosts at the
same time.  This will confuse everyone, including yourself.

\subsection{System administration}

Sometimes we make changes to Hoon or Arvo (we never make changes
to Nock) and you need to update your ship.

There are two steps to updating.  You need to get the new files,
and you need to install them.  To get them:

\begin{framed_shaded}
\begin{Verbatim}[fontsize=\relsize{-2.5},fontseries=b,commandchars=\\\{\}]
\PYZti{}waclux\PYZhy{}tomwyc/try=\PYZgt{} :update
: /\PYZti{}waclux\PYZhy{}tomwyc/arvo/2/hoon/hoon
: /\PYZti{}waclux\PYZhy{}tomwyc/arvo/2/dill/hoon
: /\PYZti{}waclux\PYZhy{}tomwyc/arvo/2/batz/hoon
\end{Verbatim}
\end{framed_shaded}

To install them (the simplest, slowest, most general way):

\begin{framed_shaded}
\begin{Verbatim}[fontsize=\relsize{-2.5},fontseries=b,commandchars=\\\{\}]
\PYZti{}waclux\PYZhy{}tomwyc/try=\PYZgt{} :reset

\PYZpc{}reset\PYZhy{}start
\PYZpc{}reset\PYZhy{}parsed
\PYZpc{}reset\PYZhy{}compiled
\PYZpc{}hoon\PYZhy{}load
[\PYZpc{}tang /\PYZti{}waclux\PYZhy{}tomwyc/arvo/\PYZti{}2013.11.26..20.29.15..090f/zuse \PYZti{}tirnux\PYZhy{}latwex]
[\PYZpc{}vane \PYZpc{}a /\PYZti{}waclux\PYZhy{}tomwyc/arvo/\PYZti{}2013.11.26..20.29.15..090f/ames \PYZti{}tolryn\PYZhy{}watret]
[\PYZpc{}vane \PYZpc{}b /\PYZti{}waclux\PYZhy{}tomwyc/arvo/\PYZti{}2013.11.26..20.29.15..090f/batz \PYZti{}donfex\PYZhy{}ladsem]
[\PYZpc{}vane \PYZpc{}c /\PYZti{}waclux\PYZhy{}tomwyc/arvo/\PYZti{}2013.11.26..20.29.15..090f/clay \PYZti{}picsug\PYZhy{}mitref]
[\PYZpc{}vane \PYZpc{}d /\PYZti{}waclux\PYZhy{}tomwyc/arvo/\PYZti{}2013.11.26..20.29.15..090f/dill \PYZti{}dilpex\PYZhy{}laptug]
[\PYZpc{}vane \PYZpc{}e /\PYZti{}waclux\PYZhy{}tomwyc/arvo/\PYZti{}2013.11.26..20.29.15..090f/eyre \PYZti{}forbur\PYZhy{}disben]
\end{Verbatim}
\end{framed_shaded}

All of your state, including running tasks, will be unchanged.

\subsection{Chat}

Okay, fine.  You're a long way from being an Arvo ninja.  But---
you're ready for the two most important uses of Urbit right now.
One, coding.  Two, chatting.

To start coding, read the next chapter.  To start chatting,
simply type

\begin{framed_shaded}
\begin{Verbatim}[fontsize=\relsize{-2.5},fontseries=b,commandchars=\\\{\}]
\PYZti{}waclux\PYZhy{}tomwyc/try=\PYZgt{} :chat
\PYZam{}
\end{Verbatim}
\end{framed_shaded}

and type \kode{?} for help.